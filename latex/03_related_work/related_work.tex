%==============================================================================
\section{Related Work}

Hexapod mobility is a well solved problem. 
% ToDo: fill in different papers here

\subsection{ Previous Projects }

Wanda has been used as a project platform by the author since summer 2021 for graduate course work semester projects. Each semester, new capabilites were added to the system. A summary of this timeline is as follows.

\begin{itemize}
    \item Summer 2021: the robot was first assembled and calibrated. Simple kinematics were established, the operating system was set up, and the hardware and software were able to communicate.
    \item Fall 2021: a simple forward wave gait was implemented on the robot to allow it to walk forward in a straight line. The sensors were used to detect its environment. 
    \item Winter 2022: all software was ported over to ROS1 Noetic. A more robust, omni-directional walking gait was implemented. The camera was used to identify and follow simple blue targets.
    \item Summer 2022: manual teleoperation was introduced using an Xbox controller. This allowed direct user control of the walking direction, the seeker angles, and all 6 degree of freedom (DOF) of the body pose, with very simple and limiting constraints on the commanded pose.
    \item Fall 2022: distributed processing was set up to offload much of the computational processing to a powerful desktop, allowing the embedded computer to run under less load. ORB\_SLAM3 \cite{orbslam} was implemented on the hardware.
\end{itemize}

The robot uses a forward wave gait to walk. The system receives a commanded velocity twist vector defined in the ground frame below the robot, $\xi_g$, with linear $x$ and $y$ and angular yaw $\psi$ components. A gait trajectory algorithm increments the position of each foot by translating and inverting this twist into each foots current position and integrating with the time step $dt$. Transition between stride and support phase is scheduled by incrementing a master stride phase based on how far all of the feet move away from the center of their workspaces. When a foot enters stride phase, the foot is picked up off the ground and moved back to the center of its workspace relative to the ground frame. This motion is executed as a polynomial trajectory with zero initial and final velocity and acceleration in the $\hat{z}$ direction. 

% ToDo: put a picture of this type of motion here?
\begin{figure}[H]
    \centerline{\includegraphics[scale=0.1]{./03_related_work/figures/old_trajectory.png}}
    \caption{Foot trajectory for simple forward walking gait}
    \label{fig:old_gait}
\end{figure}

The pose of the body can be independently controlled from the feet trajectories. This pose is calculated relative to the ground frame under the robot. The desired position of each foot and the body pose are then both passed to a simple inverse kinematics solver, which returns the commanded joint angles. Thes angles are then sent to the servos to actuate the robot. This scheme can be seen in Figure \ref{fig:old_gait_block_diagram}.

\begin{figure}[H]
    \centerline{\includegraphics[scale=0.1]{./03_related_work/figures/previous_kinematic_chain.png}}
    \caption{Previous walking gait design block diagram}
    \label{fig:old_gait_block_diagram}
\end{figure}
