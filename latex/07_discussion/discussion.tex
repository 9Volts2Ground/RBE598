%==============================================================================

\section{ Discussion }

Analytically constraining the body pose was a safety-critical prerequisite to implementing a classical body pose controller on the robot. The approach presented allowed the body pose and feet to maintian their maximum workspace values while synergistically updated together. After implementing this system, inverse kinematic errors were not observed. This safety check provided confidence in implementing the body pose PID controller, allowing the controller to move the pose as necessary without exceeding kinematic limitations. 

Global motion planning was less successful on this system. While the path planning worked effectively and was demonstrated in Rviz, this assumed that the odometry solution returned the truth. In reality, the EKF often drifted significantly from the true pose of the robot. This is largely due to foot slippage when walking. The robot was tested on carpet, and the small footprint of each leg did not provide enough traction to mantain solid contact consistently. The noisy IMU was also to blame, since it was not reliable enough to help correct the EKF much when this drifting occured. 

Measured external clutter was another factor in preventing the physical robot from consistently reaching its commanded goals. Despite filtering the range sensor observables, the robot often returned obstacles that did not appear to be physically present. This often took the form of ground clutter. While efforts were taken to prevent ground clutter return (pointing the seeker up, and checking if the edge of the sensor beam width collides with the ground plane at that range), this needs further refinement. This often resulted in the robot detecting a barrier of obstacles blocking the path that was not present in reality, and the robot could not create a solution to navigate to the goal. While an improved filter for this sensor would likely improve performance, this was beyond the scope of this project. 

Ultimately, better sensor hardware and filtering is needed to reliably implement global motion planning and following on this type of hexapod robot. Absolute position sensors like GPS would be useful in localizing the robot, although that may not be useful given the indoor setting and the small scale of motion. Higher fidelity range sensors like a lidar could instead be used to implement a SLAM algorithm like $amcl$ \cite{amcl}. The embedded camera could also be used to contribute to localization, or at least obstacle detection, but this was also beyond the scope of this project.

The hardware design of the feet also contributed somewhat to this poor localization. The leg links are made of a slick, laser cut acrylic. The tibia narrows to a fine point, providing minimal foot contact with the ground. When walking across carpet, this small footprint does not provide enough traction to prevent slippage. A larger footprint with a higher friction coefficient (perhapse using something like a rubber coating) could help minimize this slippage that contributes to EKF drifting. While this would be benificial, dead reckoning odometry drifting can ultimately never be completely eliminated without global localization techniques, and better sensing would still be necessary for the best solution.  
